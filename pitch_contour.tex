% pitch_contour.tex
% 
% This file contains TikX (http://sourceforge.net/projects/pgf) code to
% enable the drawing of pitch levels and contours over text.
% 
% Save the file and then call it from your document as follows:
% \usepackage{tikz}
% \usetikzlibrary{fit}
% % pitch_contour.tex
% 
% This file contains TikX (http://sourceforge.net/projects/pgf) code to
% enable the drawing of pitch levels and contours over text.
% 
% Save the file and then call it from your document as follows:
% \usepackage{tikz}
% \usetikzlibrary{fit}
% % pitch_contour.tex
% 
% This file contains TikX (http://sourceforge.net/projects/pgf) code to
% enable the drawing of pitch levels and contours over text.
% 
% Save the file and then call it from your document as follows:
% \usepackage{tikz}
% \usetikzlibrary{fit}
% % pitch_contour.tex
% 
% This file contains TikX (http://sourceforge.net/projects/pgf) code to
% enable the drawing of pitch levels and contours over text.
% 
% Save the file and then call it from your document as follows:
% \usepackage{tikz}
% \usetikzlibrary{fit}
% \input{path/to/pitch_contour}
% 
% Copyright Mark Wibrow 2013.
% 
% This program is free software: you can redistribute it and/or modify
% it under the terms of the GNU General Public License or the GNU
% Affero General Public License as published by the Free Software
% Foundation, either version 3 of the License, or (at your option)
% any later version.
% 
% This program is distributed in the hope that it will be useful,
% but WITHOUT ANY WARRANTY; without even the implied warranty of
% MERCHANTABILITY or FITNESS FOR A PARTICULAR PURPOSE.  See the
% GNU General Public License for more details.
% 
% You should have received a copy of the GNU General Public License
% and the GNU Affero General Public License along with this program.
% If not, see <http://www.gnu.org/licenses/>.

\newdimen\contourraise
\newdimen\contourspacetokenwidth
\newdimen\contourtokenkern
\newcount\lasttokennumber
\newcount\currenttokennumber
\newcount\contourmarkcount
\newcount\contourtokenunderlinestate
\newbox\contourbox
\makeatletter

\tikzset{
    tight fit/.style={
        inner sep=0pt,
        outer sep=0pt,
    },
    %
    %
    % How far above the reference anchor of the text,
    contour raise/.code=\pgfmathsetlength\contourraise{#1},
    contour reference anchor/.store in=\contourreferenceanchor,
    contour reference anchor=base east,
    % The `scale' for the values in the contour height specification
    contour scale/.store in=\contourscale,
    contour scale=3pt,
    % The prefix for the contour marks.
    contour mark prefix/.store in=\contourmarkprefix,
    contour mark prefix=contour,
    % The style for the contour path
    contour/.style={
        draw, 
        rounded corners=1ex,
    },
    % The style for the token nodes
    every contour token/.style={
        anchor=base west, 
        tight fit,
    },
    contour underline/.style={
        draw
    },
    % The character to insert a mark (use with care)
    contour mark character/.store in=\contourmarkchar,
    contour mark character=|,
    % Want to change the code for contour marks? Use this key.
    contour mark code/.store in=\contourmarkcode,
    % Want to change the code for tokens? Use this key.
    contour token code/.store in=\contourtokencode,
    % Want to change the code for drawing the contour? Use this  key.
    contour code/.store in=\contourcode,
    %
    % Default stuff
    contour mark code={%
        \coordinate (\contourmarkprefix-\the\contourmarkcount)
          at ([yshift=\contourraise, y=\contourscale,               
          shift={(0,\currentcontourheight)}]token-\the\currenttokennumber.\contourreferenceanchor);
    },
    contour token code={%
        \node [every contour token/.try] at 
        ([xshift=\contourtokenkern]token-\the\lasttokennumber.base east) 
            (token-\the\currenttokennumber) {\token};
    },
    contour code={
        \draw [contour] (\contourmarkprefix-1)
            \foreach \y in {2,...,\the\contourmarkcount}{ -- 
                    (\contourmarkprefix-\y) };                  
    },
    contour marks/.style={
        contour mark list={#1},
        contour code={
             \draw [y=\contourscale, contour] \contourpath;                  
         },
         contour mark code={%
            \coordinate (@a) at ([yshift=\contourraise]token-\the\currenttokennumber.base west);
            \coordinate (@b) at ([yshift=\contourraise]token-\the\currenttokennumber.base east);
            \node [tight fit, fit={(@a) (@b)}] (\contourmarkprefix-\the\contourmarkcount) {};
        },
    },
    % Don't draw the contour.
    tokens only/.style={
        contour code={}
    },
    %
    % Only draw the contour (but the space is still used for the tokens)
    contour only/.style={
        every contour token/.append style={
            execute at begin node={\setbox\contourbox=\hbox\bgroup},
            execute at end node=\egroup\phantom{\box\contourbox}%
        },
        underline/.style={
            draw=none
        }
    },
    %
    % Make tokens follow the contour marks.
    tokens follow contour/.style={
        tokens only,
        contour token code={%
            \node [every contour token/.try, y=\contourscale] at 
                ([xshift=\contourtokenkern]token-\the\lasttokennumber.base east |- 
                0,\currentcontourheight) 
                (token-\the\currenttokennumber) {\token};
        },
    },
    % What style to use when drawing underline
    underline/.style={
        draw
    },
    % The underline is drawn along the south side of a node which 
    % takes this style.
    underline token/.style={
        inner ysep=1pt
    },
    % When grouping tokens (e.g., for putting box around)
    % this style is applied to a node that is fitted around the group
    token group/.style={
        inner xsep=1pt,
        inner ysep=2pt,
        rounded corners=2pt
    },
    % Draw boxes around tokens groups.
    box tokens/.style={
        token group/.append style={
            draw
        }
    },  
    % Change the width of the spaces.
    space token width/.code=\pgfmathsetlength\contourspacetokenwidth{#1},
    space token width=0.125cm,
    contour mark list/.store in=\@contourmarklist%
}

\def\at@{@}

\let\@contourmarklist=\@empty

\def\contour{%
    \pgfutil@ifnextchar[{\contour@opts}{\contour@opts[]}}
\def\contour@opts[#1]{%
    \pgfutil@ifnextchar x{\contour@@opts[#1]}{\contour@@opts[#1]}}
\def\contour@@opts[#1]#2;{%
    \begin{scope}[#1]
        \coordinate (token-0);
        \currenttokennumber=0\relax%
        \lasttokennumber=0\relax%
        \contourmarkcount=0\relax%
        \def\lastcontourheight{0}%
        \contourtokenunderlinestate=0\relax%
        \let\lastcontourtoken=\relax%
        \contourtokenkern=0pt\relax%
        \def\contourpath{}%
        \@contour#2@%
}

% Must check for spaces
\def\@contour{\futurelet\@token\@checkforspace}

\def\@uscore{_}
\def\@checkforspace{%
    \ifx\@token\pgfutil@sptoken%
        \let\@next=\@replacespace%
    \else%
        \if\@token\contourmarkchar%
            \let\@next=\@contour@insertmark
        \else%
            \if\@token\@uscore
                \let\@next=\@contourtoggleunderline%
            \else%
                \let\@next=\@@contour%
            \fi%
        \fi%
    \fi%
    \@next%
}

\def\@contourtoggleunderline#1{%
    \advance\contourtokenunderlinestate by1\relax
    \ifnum\contourtokenunderlinestate>3\relax%
        \contourtokenunderlinestate=0\relax%
    \fi%
    \@contour%
}

\def\@contour@insertmark{%
    \afterassignment\@@contour@insertmark\let\@token=%
}

\def\@@contour@insertmark{%
    \futurelet\@token\@@@contour@insertmark}%

\def\@@@contour@insertmark{%
    \if\@token[%
        \let\@next=\@@@@contour@insertmark%
    \else%
        \let\currentcontourheight=\lastcontourheight%
        \let\@next=\@@@@@contour@insertmark%
    \fi%
    \@next%
}

\def\@@@@contour@insertmark[#1]{%
    \def\@tmp{#1}%
    \ifx\@tmp\@empty%
        \let\currentcontourheight=\lastcontourheight%
    \else%
        \def\currentcontourheight{#1}%
    \fi%
    \@@@@@contour@insertmark}

\def\@@@@@contour@insertmark{%
    \advance\contourmarkcount by1\relax%
     % Code for inserting mark
    \contourmarkcode%
    \let\lastcontourheight=\currentcontourheight%
    \@contour}

\def\contourspacetoken{{\hbox to \contourspacetokenwidth{\hfill}}}

\def\@replacespace#1{%
    \@contour\contourspacetoken#1%
}

\def\@@countour@afterlatenode{%
    \pgf@x=\pgfpositionnodelatermaxx\relax%
    \advance\pgf@x by-\pgfpositionnodelaterminx\relax%
    \global\edef\@contournodewidth{\the\pgf@x}%
}

\def\@@contour#1{%
    \def\@token{#1}%
    \if\@token\at@%
        \@contourdounderline%
        \pgfutil@ifundefined{pgf@sh@ns@tokengroup}{}{%
            \node [tight fit, fit={(tokengroup)}, token group/.try] {};
            \global\let\pgf@sh@ns@tokengroup=\relax%
        }%
        \let\@next=\@@@contour%
    \else%
        \lasttokennumber=\currenttokennumber%
        \advance\currenttokennumber by1%
        \contourtokenkern=0pt\relax%
        \ifnum\currenttokennumber>1\relax%
            %
            % Take care of kerning.
            % 
            % First get the width of the last and current token in the same hbox.
            %
            \let\pgfpositionnodelaterbox=\contourbox
            \pgfpositionnodelater\@@countour@afterlatenode%
            \def\token{\lastcontourtoken\@token}%
            \begingroup%
                \tikzset{every contour token/.append style={tight fit}}%
                \contourtokencode%
            \endgroup%
            \let\@contourkerntmp=\@contournodewidth%
            % 
            % Now subtract the width of last and current token in separate boxes.
            %
            \def\token{\hbox{\lastcontourtoken}\hbox{\@token}}%
            \begingroup%
                    \tikzset{every contour token/.append style={tight fit}}%
                    \contourtokencode%
            \endgroup%
            \pgfmathsetlength\contourtokenkern{\@contourkerntmp-\@contournodewidth}%
            \pgfpositionnodelater\relax%
        \fi%
        %
        % OK, now actually typeset the current token
        %
        \let\token=\@token%
        \contourtokencode%
        \let\lastcontourtoken=\token%
        % Manage underline state
        \@contourdounderline%
        \def\@@token{\contourspacetoken}%
        \ifx\@token\@@token%
            \pgfutil@ifundefined{pgf@sh@ns@tokengroup}{}{%
                \pgfutil@ifundefined{pgf@sh@ns@underline}{}{%
                    \node [tight fit, fit={(tokengroup) (underline)}] 
                    (tokengroup) 
                {};}%
                \node [tight fit, fit={(tokengroup)}, token group/.try] {};
                \global\let\pgf@sh@ns@tokengroup=\relax%
            }%
        \else
            \pgfutil@ifundefined{pgf@sh@ns@tokengroup}{%
                \node [tight fit, 
                fit={(token-\the\currenttokennumber)}] 
                (tokengroup) {};
            }{%
                \node [tight fit, 
                fit={(token-\the\currenttokennumber) 
                (tokengroup)}] 
                (tokengroup){};
            }%
        \fi%
        \let\@next=\@contour
        %
    \fi%
    \@next%
}

\def\@contourdounderline{%
    \ifcase\contourtokenunderlinestate%
     \or
         \node [tight fit, fit={(token-\the\currenttokennumber)}] 
         (underline) {};
         \contourtokenunderlinestate=2\relax%
     \or%
            \node [tight fit,fit={(token-\the\currenttokennumber) (underline)}]
            (underline) {};
     \or%
            \node [tight fit, fit={(underline)}, underline token/.try] 
            (underline) {};
         \draw [underline/.try]
                    (underline.south west) -- (underline.south east);
            \pgfutil@ifundefined{pgf@sh@ns@tokengroup}{}{%
                 \node [tight fit, fit={(tokengroup) (underline)}] 
                 (tokengroup) {};%
                 \node [tight fit, fit={(tokengroup)}, token group/.try] {};
                 \global\let\pgf@sh@ns@tokengroup=\relax%
                 \global\let\pgf@sh@ns@underline=\relax%
             }
         \contourtokenunderlinestate=0\relax
     \fi%
}
\def\@@@contour{%
    \ifx\@contourmarklist\@empty%
    \else%
        \@contourdolist%
    \fi%
    \ifnum\contourmarkcount>1
        % Code for drawing contour
        \contourcode%
    \fi%
    \end{scope}%
    \ignorespaces%
}

\def\@contourstackpop{%
    \let\@contourstackitem=\@empty%
    \ifx\@contourstack\@empty%
    \else%
        \expandafter\@@contourstackpop\@contourstack\@@contourstackpop%
    \fi%
}

\def\@@contourstackpop#1#2\@@contourstackpop{%
    \def\@contourstackitem{#1}%
    \ifx\@contourstackitem\@empty%
        \def\@contourstackitem{#2}%
        \let\@contourstack=\@empty%
    \else%
        \def\@contourstack{#2}%
    \fi%
}

\def\@contourdolist{%
    \let\@contourstack=\@contourmarklist%
    \let\@contourstacklastitem=\@empty%
    \let\contourpath=\@empty%
    \edef\contourtotaltokens{\the\currenttokennumber}%
    \currenttokennumber=0\relax%
    \contourmarkcount=0\relax%
    \@@contourdolist%
}

\def\@@contourdolist{%
    \@contourstackpop%
    \advance\currenttokennumber by1\relax%
    \ifx\@contourstackitem\@empty%
        \let\@next=\relax%
    \else%
        \expandafter\ifx\csname contourcontourpathcommand@\@contourstackitem @\endcsname\relax%
        \else%
            \advance\contourmarkcount by1\relax%
            \let\currentcontourheight=\@contourstackitem%
            \contourmarkcode%
            \def\contourmarkstart{\contourmarkprefix-\the\contourmarkcount.west}%
            \def\contourmarkend{\contourmarkprefix-\the\contourmarkcount.east}%         
            \edef\contourpath{\contourpath \csname contourcontourpathcommand@\@contourstackitem @\endcsname}%
        \fi%
        \let\@next=\@@contourdolist%
        \let\@contourstacklastitem=\@contourstackitem%
    \fi
    \@next%
}

% \contourcontourpathcommand{<symbol>}{<contour path command code>}
% \contourmarkstart and \contourmarkend are setup as the
% left and right points of the charactor at zero contour height.
\def\contourcontourpathcommand#1{\expandafter\def\csname contourcontourpathcommand@#1@\endcsname}

% \contourmark{<symbol>}{<mark start height>}{<mark end height>}

\def\contourmark#1#2#3{%
    \contourcontourpathcommand{#1}{([shift={(0,#2)}]\contourmarkstart) -- ([shift={(0,#3)}]\contourmarkend)}
}

\makeatother

% Set up pitchmarks for inline and multiline depiction of tonal pitchlevels. 
% Other characters can be used to create a pitchmark, eg: < > ( ) - _ = '

\contourcontourpathcommand{.}{}

\contourmark{0}{0}{0}
\contourmark{1}{1}{1}
\contourmark{2}{2}{2}
\contourmark{3}{3}{3}
\contourmark{4}{4}{4}
\contourmark{5}{5}{5}
\contourmark{+}{0}{1}
\contourmark{,}{0}{2}
\contourmark{>}{0}{4}

% Sample pitchmarks
% Rise and fall
\contourcontourpathcommand{!}{
    (\contourmarkstart) .. controls ++(0,2) and ++(0,2) .. (\contourmarkend)
}

% Fall and rise - shift specifies the vertical height of the startpoint and endpoint
\contourcontourpathcommand{@}{
    ([shift={(0,2)}]\contourmarkstart) .. controls ++(0,-2) and ++(0,-2) .. ([shift={(0,2)}]\contourmarkend)
}

% Glide up (concave)
\contourcontourpathcommand{?}{
    (\contourmarkstart) .. controls ++(0,-1) and ++(0,-1) .. ([shift={(0,2)}]\contourmarkend)
}

% Glide up (convex)
\contourcontourpathcommand{:}{
    (\contourmarkstart) .. controls ++(0,1) and ++(0,1) .. ([shift={(0,2)}]\contourmarkend)
}

% Glide down (concave)
\contourcontourpathcommand{;}{
    ([shift={(0,2)}]\contourmarkstart) .. controls ++(0,-1) and ++(0,-1) .. (\contourmarkend)
}

% Glide down (convex)
\contourcontourpathcommand{^}{
    ([shift={(0,2)}]\contourmarkstart) .. controls ++(0,1) and ++(0,1) .. (\contourmarkend)
}

% Continue previous path
\contourcontourpathcommand{|}{
    -- ([shift={(0, 3)}]\contourmarkend)
}



% 
% Copyright Mark Wibrow 2013.
% 
% This program is free software: you can redistribute it and/or modify
% it under the terms of the GNU General Public License or the GNU
% Affero General Public License as published by the Free Software
% Foundation, either version 3 of the License, or (at your option)
% any later version.
% 
% This program is distributed in the hope that it will be useful,
% but WITHOUT ANY WARRANTY; without even the implied warranty of
% MERCHANTABILITY or FITNESS FOR A PARTICULAR PURPOSE.  See the
% GNU General Public License for more details.
% 
% You should have received a copy of the GNU General Public License
% and the GNU Affero General Public License along with this program.
% If not, see <http://www.gnu.org/licenses/>.

\newdimen\contourraise
\newdimen\contourspacetokenwidth
\newdimen\contourtokenkern
\newcount\lasttokennumber
\newcount\currenttokennumber
\newcount\contourmarkcount
\newcount\contourtokenunderlinestate
\newbox\contourbox
\makeatletter

\tikzset{
    tight fit/.style={
        inner sep=0pt,
        outer sep=0pt,
    },
    %
    %
    % How far above the reference anchor of the text,
    contour raise/.code=\pgfmathsetlength\contourraise{#1},
    contour reference anchor/.store in=\contourreferenceanchor,
    contour reference anchor=base east,
    % The `scale' for the values in the contour height specification
    contour scale/.store in=\contourscale,
    contour scale=3pt,
    % The prefix for the contour marks.
    contour mark prefix/.store in=\contourmarkprefix,
    contour mark prefix=contour,
    % The style for the contour path
    contour/.style={
        draw, 
        rounded corners=1ex,
    },
    % The style for the token nodes
    every contour token/.style={
        anchor=base west, 
        tight fit,
    },
    contour underline/.style={
        draw
    },
    % The character to insert a mark (use with care)
    contour mark character/.store in=\contourmarkchar,
    contour mark character=|,
    % Want to change the code for contour marks? Use this key.
    contour mark code/.store in=\contourmarkcode,
    % Want to change the code for tokens? Use this key.
    contour token code/.store in=\contourtokencode,
    % Want to change the code for drawing the contour? Use this  key.
    contour code/.store in=\contourcode,
    %
    % Default stuff
    contour mark code={%
        \coordinate (\contourmarkprefix-\the\contourmarkcount)
          at ([yshift=\contourraise, y=\contourscale,               
          shift={(0,\currentcontourheight)}]token-\the\currenttokennumber.\contourreferenceanchor);
    },
    contour token code={%
        \node [every contour token/.try] at 
        ([xshift=\contourtokenkern]token-\the\lasttokennumber.base east) 
            (token-\the\currenttokennumber) {\token};
    },
    contour code={
        \draw [contour] (\contourmarkprefix-1)
            \foreach \y in {2,...,\the\contourmarkcount}{ -- 
                    (\contourmarkprefix-\y) };                  
    },
    contour marks/.style={
        contour mark list={#1},
        contour code={
             \draw [y=\contourscale, contour] \contourpath;                  
         },
         contour mark code={%
            \coordinate (@a) at ([yshift=\contourraise]token-\the\currenttokennumber.base west);
            \coordinate (@b) at ([yshift=\contourraise]token-\the\currenttokennumber.base east);
            \node [tight fit, fit={(@a) (@b)}] (\contourmarkprefix-\the\contourmarkcount) {};
        },
    },
    % Don't draw the contour.
    tokens only/.style={
        contour code={}
    },
    %
    % Only draw the contour (but the space is still used for the tokens)
    contour only/.style={
        every contour token/.append style={
            execute at begin node={\setbox\contourbox=\hbox\bgroup},
            execute at end node=\egroup\phantom{\box\contourbox}%
        },
        underline/.style={
            draw=none
        }
    },
    %
    % Make tokens follow the contour marks.
    tokens follow contour/.style={
        tokens only,
        contour token code={%
            \node [every contour token/.try, y=\contourscale] at 
                ([xshift=\contourtokenkern]token-\the\lasttokennumber.base east |- 
                0,\currentcontourheight) 
                (token-\the\currenttokennumber) {\token};
        },
    },
    % What style to use when drawing underline
    underline/.style={
        draw
    },
    % The underline is drawn along the south side of a node which 
    % takes this style.
    underline token/.style={
        inner ysep=1pt
    },
    % When grouping tokens (e.g., for putting box around)
    % this style is applied to a node that is fitted around the group
    token group/.style={
        inner xsep=1pt,
        inner ysep=2pt,
        rounded corners=2pt
    },
    % Draw boxes around tokens groups.
    box tokens/.style={
        token group/.append style={
            draw
        }
    },  
    % Change the width of the spaces.
    space token width/.code=\pgfmathsetlength\contourspacetokenwidth{#1},
    space token width=0.125cm,
    contour mark list/.store in=\@contourmarklist%
}

\def\at@{@}

\let\@contourmarklist=\@empty

\def\contour{%
    \pgfutil@ifnextchar[{\contour@opts}{\contour@opts[]}}
\def\contour@opts[#1]{%
    \pgfutil@ifnextchar x{\contour@@opts[#1]}{\contour@@opts[#1]}}
\def\contour@@opts[#1]#2;{%
    \begin{scope}[#1]
        \coordinate (token-0);
        \currenttokennumber=0\relax%
        \lasttokennumber=0\relax%
        \contourmarkcount=0\relax%
        \def\lastcontourheight{0}%
        \contourtokenunderlinestate=0\relax%
        \let\lastcontourtoken=\relax%
        \contourtokenkern=0pt\relax%
        \def\contourpath{}%
        \@contour#2@%
}

% Must check for spaces
\def\@contour{\futurelet\@token\@checkforspace}

\def\@uscore{_}
\def\@checkforspace{%
    \ifx\@token\pgfutil@sptoken%
        \let\@next=\@replacespace%
    \else%
        \if\@token\contourmarkchar%
            \let\@next=\@contour@insertmark
        \else%
            \if\@token\@uscore
                \let\@next=\@contourtoggleunderline%
            \else%
                \let\@next=\@@contour%
            \fi%
        \fi%
    \fi%
    \@next%
}

\def\@contourtoggleunderline#1{%
    \advance\contourtokenunderlinestate by1\relax
    \ifnum\contourtokenunderlinestate>3\relax%
        \contourtokenunderlinestate=0\relax%
    \fi%
    \@contour%
}

\def\@contour@insertmark{%
    \afterassignment\@@contour@insertmark\let\@token=%
}

\def\@@contour@insertmark{%
    \futurelet\@token\@@@contour@insertmark}%

\def\@@@contour@insertmark{%
    \if\@token[%
        \let\@next=\@@@@contour@insertmark%
    \else%
        \let\currentcontourheight=\lastcontourheight%
        \let\@next=\@@@@@contour@insertmark%
    \fi%
    \@next%
}

\def\@@@@contour@insertmark[#1]{%
    \def\@tmp{#1}%
    \ifx\@tmp\@empty%
        \let\currentcontourheight=\lastcontourheight%
    \else%
        \def\currentcontourheight{#1}%
    \fi%
    \@@@@@contour@insertmark}

\def\@@@@@contour@insertmark{%
    \advance\contourmarkcount by1\relax%
     % Code for inserting mark
    \contourmarkcode%
    \let\lastcontourheight=\currentcontourheight%
    \@contour}

\def\contourspacetoken{{\hbox to \contourspacetokenwidth{\hfill}}}

\def\@replacespace#1{%
    \@contour\contourspacetoken#1%
}

\def\@@countour@afterlatenode{%
    \pgf@x=\pgfpositionnodelatermaxx\relax%
    \advance\pgf@x by-\pgfpositionnodelaterminx\relax%
    \global\edef\@contournodewidth{\the\pgf@x}%
}

\def\@@contour#1{%
    \def\@token{#1}%
    \if\@token\at@%
        \@contourdounderline%
        \pgfutil@ifundefined{pgf@sh@ns@tokengroup}{}{%
            \node [tight fit, fit={(tokengroup)}, token group/.try] {};
            \global\let\pgf@sh@ns@tokengroup=\relax%
        }%
        \let\@next=\@@@contour%
    \else%
        \lasttokennumber=\currenttokennumber%
        \advance\currenttokennumber by1%
        \contourtokenkern=0pt\relax%
        \ifnum\currenttokennumber>1\relax%
            %
            % Take care of kerning.
            % 
            % First get the width of the last and current token in the same hbox.
            %
            \let\pgfpositionnodelaterbox=\contourbox
            \pgfpositionnodelater\@@countour@afterlatenode%
            \def\token{\lastcontourtoken\@token}%
            \begingroup%
                \tikzset{every contour token/.append style={tight fit}}%
                \contourtokencode%
            \endgroup%
            \let\@contourkerntmp=\@contournodewidth%
            % 
            % Now subtract the width of last and current token in separate boxes.
            %
            \def\token{\hbox{\lastcontourtoken}\hbox{\@token}}%
            \begingroup%
                    \tikzset{every contour token/.append style={tight fit}}%
                    \contourtokencode%
            \endgroup%
            \pgfmathsetlength\contourtokenkern{\@contourkerntmp-\@contournodewidth}%
            \pgfpositionnodelater\relax%
        \fi%
        %
        % OK, now actually typeset the current token
        %
        \let\token=\@token%
        \contourtokencode%
        \let\lastcontourtoken=\token%
        % Manage underline state
        \@contourdounderline%
        \def\@@token{\contourspacetoken}%
        \ifx\@token\@@token%
            \pgfutil@ifundefined{pgf@sh@ns@tokengroup}{}{%
                \pgfutil@ifundefined{pgf@sh@ns@underline}{}{%
                    \node [tight fit, fit={(tokengroup) (underline)}] 
                    (tokengroup) 
                {};}%
                \node [tight fit, fit={(tokengroup)}, token group/.try] {};
                \global\let\pgf@sh@ns@tokengroup=\relax%
            }%
        \else
            \pgfutil@ifundefined{pgf@sh@ns@tokengroup}{%
                \node [tight fit, 
                fit={(token-\the\currenttokennumber)}] 
                (tokengroup) {};
            }{%
                \node [tight fit, 
                fit={(token-\the\currenttokennumber) 
                (tokengroup)}] 
                (tokengroup){};
            }%
        \fi%
        \let\@next=\@contour
        %
    \fi%
    \@next%
}

\def\@contourdounderline{%
    \ifcase\contourtokenunderlinestate%
     \or
         \node [tight fit, fit={(token-\the\currenttokennumber)}] 
         (underline) {};
         \contourtokenunderlinestate=2\relax%
     \or%
            \node [tight fit,fit={(token-\the\currenttokennumber) (underline)}]
            (underline) {};
     \or%
            \node [tight fit, fit={(underline)}, underline token/.try] 
            (underline) {};
         \draw [underline/.try]
                    (underline.south west) -- (underline.south east);
            \pgfutil@ifundefined{pgf@sh@ns@tokengroup}{}{%
                 \node [tight fit, fit={(tokengroup) (underline)}] 
                 (tokengroup) {};%
                 \node [tight fit, fit={(tokengroup)}, token group/.try] {};
                 \global\let\pgf@sh@ns@tokengroup=\relax%
                 \global\let\pgf@sh@ns@underline=\relax%
             }
         \contourtokenunderlinestate=0\relax
     \fi%
}
\def\@@@contour{%
    \ifx\@contourmarklist\@empty%
    \else%
        \@contourdolist%
    \fi%
    \ifnum\contourmarkcount>1
        % Code for drawing contour
        \contourcode%
    \fi%
    \end{scope}%
    \ignorespaces%
}

\def\@contourstackpop{%
    \let\@contourstackitem=\@empty%
    \ifx\@contourstack\@empty%
    \else%
        \expandafter\@@contourstackpop\@contourstack\@@contourstackpop%
    \fi%
}

\def\@@contourstackpop#1#2\@@contourstackpop{%
    \def\@contourstackitem{#1}%
    \ifx\@contourstackitem\@empty%
        \def\@contourstackitem{#2}%
        \let\@contourstack=\@empty%
    \else%
        \def\@contourstack{#2}%
    \fi%
}

\def\@contourdolist{%
    \let\@contourstack=\@contourmarklist%
    \let\@contourstacklastitem=\@empty%
    \let\contourpath=\@empty%
    \edef\contourtotaltokens{\the\currenttokennumber}%
    \currenttokennumber=0\relax%
    \contourmarkcount=0\relax%
    \@@contourdolist%
}

\def\@@contourdolist{%
    \@contourstackpop%
    \advance\currenttokennumber by1\relax%
    \ifx\@contourstackitem\@empty%
        \let\@next=\relax%
    \else%
        \expandafter\ifx\csname contourcontourpathcommand@\@contourstackitem @\endcsname\relax%
        \else%
            \advance\contourmarkcount by1\relax%
            \let\currentcontourheight=\@contourstackitem%
            \contourmarkcode%
            \def\contourmarkstart{\contourmarkprefix-\the\contourmarkcount.west}%
            \def\contourmarkend{\contourmarkprefix-\the\contourmarkcount.east}%         
            \edef\contourpath{\contourpath \csname contourcontourpathcommand@\@contourstackitem @\endcsname}%
        \fi%
        \let\@next=\@@contourdolist%
        \let\@contourstacklastitem=\@contourstackitem%
    \fi
    \@next%
}

% \contourcontourpathcommand{<symbol>}{<contour path command code>}
% \contourmarkstart and \contourmarkend are setup as the
% left and right points of the charactor at zero contour height.
\def\contourcontourpathcommand#1{\expandafter\def\csname contourcontourpathcommand@#1@\endcsname}

% \contourmark{<symbol>}{<mark start height>}{<mark end height>}

\def\contourmark#1#2#3{%
    \contourcontourpathcommand{#1}{([shift={(0,#2)}]\contourmarkstart) -- ([shift={(0,#3)}]\contourmarkend)}
}

\makeatother

% Set up pitchmarks for inline and multiline depiction of tonal pitchlevels. 
% Other characters can be used to create a pitchmark, eg: < > ( ) - _ = '

\contourcontourpathcommand{.}{}

\contourmark{0}{0}{0}
\contourmark{1}{1}{1}
\contourmark{2}{2}{2}
\contourmark{3}{3}{3}
\contourmark{4}{4}{4}
\contourmark{5}{5}{5}
\contourmark{+}{0}{1}
\contourmark{,}{0}{2}
\contourmark{>}{0}{4}

% Sample pitchmarks
% Rise and fall
\contourcontourpathcommand{!}{
    (\contourmarkstart) .. controls ++(0,2) and ++(0,2) .. (\contourmarkend)
}

% Fall and rise - shift specifies the vertical height of the startpoint and endpoint
\contourcontourpathcommand{@}{
    ([shift={(0,2)}]\contourmarkstart) .. controls ++(0,-2) and ++(0,-2) .. ([shift={(0,2)}]\contourmarkend)
}

% Glide up (concave)
\contourcontourpathcommand{?}{
    (\contourmarkstart) .. controls ++(0,-1) and ++(0,-1) .. ([shift={(0,2)}]\contourmarkend)
}

% Glide up (convex)
\contourcontourpathcommand{:}{
    (\contourmarkstart) .. controls ++(0,1) and ++(0,1) .. ([shift={(0,2)}]\contourmarkend)
}

% Glide down (concave)
\contourcontourpathcommand{;}{
    ([shift={(0,2)}]\contourmarkstart) .. controls ++(0,-1) and ++(0,-1) .. (\contourmarkend)
}

% Glide down (convex)
\contourcontourpathcommand{^}{
    ([shift={(0,2)}]\contourmarkstart) .. controls ++(0,1) and ++(0,1) .. (\contourmarkend)
}

% Continue previous path
\contourcontourpathcommand{|}{
    -- ([shift={(0, 3)}]\contourmarkend)
}



% 
% Copyright Mark Wibrow 2013.
% 
% This program is free software: you can redistribute it and/or modify
% it under the terms of the GNU General Public License or the GNU
% Affero General Public License as published by the Free Software
% Foundation, either version 3 of the License, or (at your option)
% any later version.
% 
% This program is distributed in the hope that it will be useful,
% but WITHOUT ANY WARRANTY; without even the implied warranty of
% MERCHANTABILITY or FITNESS FOR A PARTICULAR PURPOSE.  See the
% GNU General Public License for more details.
% 
% You should have received a copy of the GNU General Public License
% and the GNU Affero General Public License along with this program.
% If not, see <http://www.gnu.org/licenses/>.

\newdimen\contourraise
\newdimen\contourspacetokenwidth
\newdimen\contourtokenkern
\newcount\lasttokennumber
\newcount\currenttokennumber
\newcount\contourmarkcount
\newcount\contourtokenunderlinestate
\newbox\contourbox
\makeatletter

\tikzset{
    tight fit/.style={
        inner sep=0pt,
        outer sep=0pt,
    },
    %
    %
    % How far above the reference anchor of the text,
    contour raise/.code=\pgfmathsetlength\contourraise{#1},
    contour reference anchor/.store in=\contourreferenceanchor,
    contour reference anchor=base east,
    % The `scale' for the values in the contour height specification
    contour scale/.store in=\contourscale,
    contour scale=3pt,
    % The prefix for the contour marks.
    contour mark prefix/.store in=\contourmarkprefix,
    contour mark prefix=contour,
    % The style for the contour path
    contour/.style={
        draw, 
        rounded corners=1ex,
    },
    % The style for the token nodes
    every contour token/.style={
        anchor=base west, 
        tight fit,
    },
    contour underline/.style={
        draw
    },
    % The character to insert a mark (use with care)
    contour mark character/.store in=\contourmarkchar,
    contour mark character=|,
    % Want to change the code for contour marks? Use this key.
    contour mark code/.store in=\contourmarkcode,
    % Want to change the code for tokens? Use this key.
    contour token code/.store in=\contourtokencode,
    % Want to change the code for drawing the contour? Use this  key.
    contour code/.store in=\contourcode,
    %
    % Default stuff
    contour mark code={%
        \coordinate (\contourmarkprefix-\the\contourmarkcount)
          at ([yshift=\contourraise, y=\contourscale,               
          shift={(0,\currentcontourheight)}]token-\the\currenttokennumber.\contourreferenceanchor);
    },
    contour token code={%
        \node [every contour token/.try] at 
        ([xshift=\contourtokenkern]token-\the\lasttokennumber.base east) 
            (token-\the\currenttokennumber) {\token};
    },
    contour code={
        \draw [contour] (\contourmarkprefix-1)
            \foreach \y in {2,...,\the\contourmarkcount}{ -- 
                    (\contourmarkprefix-\y) };                  
    },
    contour marks/.style={
        contour mark list={#1},
        contour code={
             \draw [y=\contourscale, contour] \contourpath;                  
         },
         contour mark code={%
            \coordinate (@a) at ([yshift=\contourraise]token-\the\currenttokennumber.base west);
            \coordinate (@b) at ([yshift=\contourraise]token-\the\currenttokennumber.base east);
            \node [tight fit, fit={(@a) (@b)}] (\contourmarkprefix-\the\contourmarkcount) {};
        },
    },
    % Don't draw the contour.
    tokens only/.style={
        contour code={}
    },
    %
    % Only draw the contour (but the space is still used for the tokens)
    contour only/.style={
        every contour token/.append style={
            execute at begin node={\setbox\contourbox=\hbox\bgroup},
            execute at end node=\egroup\phantom{\box\contourbox}%
        },
        underline/.style={
            draw=none
        }
    },
    %
    % Make tokens follow the contour marks.
    tokens follow contour/.style={
        tokens only,
        contour token code={%
            \node [every contour token/.try, y=\contourscale] at 
                ([xshift=\contourtokenkern]token-\the\lasttokennumber.base east |- 
                0,\currentcontourheight) 
                (token-\the\currenttokennumber) {\token};
        },
    },
    % What style to use when drawing underline
    underline/.style={
        draw
    },
    % The underline is drawn along the south side of a node which 
    % takes this style.
    underline token/.style={
        inner ysep=1pt
    },
    % When grouping tokens (e.g., for putting box around)
    % this style is applied to a node that is fitted around the group
    token group/.style={
        inner xsep=1pt,
        inner ysep=2pt,
        rounded corners=2pt
    },
    % Draw boxes around tokens groups.
    box tokens/.style={
        token group/.append style={
            draw
        }
    },  
    % Change the width of the spaces.
    space token width/.code=\pgfmathsetlength\contourspacetokenwidth{#1},
    space token width=0.125cm,
    contour mark list/.store in=\@contourmarklist%
}

\def\at@{@}

\let\@contourmarklist=\@empty

\def\contour{%
    \pgfutil@ifnextchar[{\contour@opts}{\contour@opts[]}}
\def\contour@opts[#1]{%
    \pgfutil@ifnextchar x{\contour@@opts[#1]}{\contour@@opts[#1]}}
\def\contour@@opts[#1]#2;{%
    \begin{scope}[#1]
        \coordinate (token-0);
        \currenttokennumber=0\relax%
        \lasttokennumber=0\relax%
        \contourmarkcount=0\relax%
        \def\lastcontourheight{0}%
        \contourtokenunderlinestate=0\relax%
        \let\lastcontourtoken=\relax%
        \contourtokenkern=0pt\relax%
        \def\contourpath{}%
        \@contour#2@%
}

% Must check for spaces
\def\@contour{\futurelet\@token\@checkforspace}

\def\@uscore{_}
\def\@checkforspace{%
    \ifx\@token\pgfutil@sptoken%
        \let\@next=\@replacespace%
    \else%
        \if\@token\contourmarkchar%
            \let\@next=\@contour@insertmark
        \else%
            \if\@token\@uscore
                \let\@next=\@contourtoggleunderline%
            \else%
                \let\@next=\@@contour%
            \fi%
        \fi%
    \fi%
    \@next%
}

\def\@contourtoggleunderline#1{%
    \advance\contourtokenunderlinestate by1\relax
    \ifnum\contourtokenunderlinestate>3\relax%
        \contourtokenunderlinestate=0\relax%
    \fi%
    \@contour%
}

\def\@contour@insertmark{%
    \afterassignment\@@contour@insertmark\let\@token=%
}

\def\@@contour@insertmark{%
    \futurelet\@token\@@@contour@insertmark}%

\def\@@@contour@insertmark{%
    \if\@token[%
        \let\@next=\@@@@contour@insertmark%
    \else%
        \let\currentcontourheight=\lastcontourheight%
        \let\@next=\@@@@@contour@insertmark%
    \fi%
    \@next%
}

\def\@@@@contour@insertmark[#1]{%
    \def\@tmp{#1}%
    \ifx\@tmp\@empty%
        \let\currentcontourheight=\lastcontourheight%
    \else%
        \def\currentcontourheight{#1}%
    \fi%
    \@@@@@contour@insertmark}

\def\@@@@@contour@insertmark{%
    \advance\contourmarkcount by1\relax%
     % Code for inserting mark
    \contourmarkcode%
    \let\lastcontourheight=\currentcontourheight%
    \@contour}

\def\contourspacetoken{{\hbox to \contourspacetokenwidth{\hfill}}}

\def\@replacespace#1{%
    \@contour\contourspacetoken#1%
}

\def\@@countour@afterlatenode{%
    \pgf@x=\pgfpositionnodelatermaxx\relax%
    \advance\pgf@x by-\pgfpositionnodelaterminx\relax%
    \global\edef\@contournodewidth{\the\pgf@x}%
}

\def\@@contour#1{%
    \def\@token{#1}%
    \if\@token\at@%
        \@contourdounderline%
        \pgfutil@ifundefined{pgf@sh@ns@tokengroup}{}{%
            \node [tight fit, fit={(tokengroup)}, token group/.try] {};
            \global\let\pgf@sh@ns@tokengroup=\relax%
        }%
        \let\@next=\@@@contour%
    \else%
        \lasttokennumber=\currenttokennumber%
        \advance\currenttokennumber by1%
        \contourtokenkern=0pt\relax%
        \ifnum\currenttokennumber>1\relax%
            %
            % Take care of kerning.
            % 
            % First get the width of the last and current token in the same hbox.
            %
            \let\pgfpositionnodelaterbox=\contourbox
            \pgfpositionnodelater\@@countour@afterlatenode%
            \def\token{\lastcontourtoken\@token}%
            \begingroup%
                \tikzset{every contour token/.append style={tight fit}}%
                \contourtokencode%
            \endgroup%
            \let\@contourkerntmp=\@contournodewidth%
            % 
            % Now subtract the width of last and current token in separate boxes.
            %
            \def\token{\hbox{\lastcontourtoken}\hbox{\@token}}%
            \begingroup%
                    \tikzset{every contour token/.append style={tight fit}}%
                    \contourtokencode%
            \endgroup%
            \pgfmathsetlength\contourtokenkern{\@contourkerntmp-\@contournodewidth}%
            \pgfpositionnodelater\relax%
        \fi%
        %
        % OK, now actually typeset the current token
        %
        \let\token=\@token%
        \contourtokencode%
        \let\lastcontourtoken=\token%
        % Manage underline state
        \@contourdounderline%
        \def\@@token{\contourspacetoken}%
        \ifx\@token\@@token%
            \pgfutil@ifundefined{pgf@sh@ns@tokengroup}{}{%
                \pgfutil@ifundefined{pgf@sh@ns@underline}{}{%
                    \node [tight fit, fit={(tokengroup) (underline)}] 
                    (tokengroup) 
                {};}%
                \node [tight fit, fit={(tokengroup)}, token group/.try] {};
                \global\let\pgf@sh@ns@tokengroup=\relax%
            }%
        \else
            \pgfutil@ifundefined{pgf@sh@ns@tokengroup}{%
                \node [tight fit, 
                fit={(token-\the\currenttokennumber)}] 
                (tokengroup) {};
            }{%
                \node [tight fit, 
                fit={(token-\the\currenttokennumber) 
                (tokengroup)}] 
                (tokengroup){};
            }%
        \fi%
        \let\@next=\@contour
        %
    \fi%
    \@next%
}

\def\@contourdounderline{%
    \ifcase\contourtokenunderlinestate%
     \or
         \node [tight fit, fit={(token-\the\currenttokennumber)}] 
         (underline) {};
         \contourtokenunderlinestate=2\relax%
     \or%
            \node [tight fit,fit={(token-\the\currenttokennumber) (underline)}]
            (underline) {};
     \or%
            \node [tight fit, fit={(underline)}, underline token/.try] 
            (underline) {};
         \draw [underline/.try]
                    (underline.south west) -- (underline.south east);
            \pgfutil@ifundefined{pgf@sh@ns@tokengroup}{}{%
                 \node [tight fit, fit={(tokengroup) (underline)}] 
                 (tokengroup) {};%
                 \node [tight fit, fit={(tokengroup)}, token group/.try] {};
                 \global\let\pgf@sh@ns@tokengroup=\relax%
                 \global\let\pgf@sh@ns@underline=\relax%
             }
         \contourtokenunderlinestate=0\relax
     \fi%
}
\def\@@@contour{%
    \ifx\@contourmarklist\@empty%
    \else%
        \@contourdolist%
    \fi%
    \ifnum\contourmarkcount>1
        % Code for drawing contour
        \contourcode%
    \fi%
    \end{scope}%
    \ignorespaces%
}

\def\@contourstackpop{%
    \let\@contourstackitem=\@empty%
    \ifx\@contourstack\@empty%
    \else%
        \expandafter\@@contourstackpop\@contourstack\@@contourstackpop%
    \fi%
}

\def\@@contourstackpop#1#2\@@contourstackpop{%
    \def\@contourstackitem{#1}%
    \ifx\@contourstackitem\@empty%
        \def\@contourstackitem{#2}%
        \let\@contourstack=\@empty%
    \else%
        \def\@contourstack{#2}%
    \fi%
}

\def\@contourdolist{%
    \let\@contourstack=\@contourmarklist%
    \let\@contourstacklastitem=\@empty%
    \let\contourpath=\@empty%
    \edef\contourtotaltokens{\the\currenttokennumber}%
    \currenttokennumber=0\relax%
    \contourmarkcount=0\relax%
    \@@contourdolist%
}

\def\@@contourdolist{%
    \@contourstackpop%
    \advance\currenttokennumber by1\relax%
    \ifx\@contourstackitem\@empty%
        \let\@next=\relax%
    \else%
        \expandafter\ifx\csname contourcontourpathcommand@\@contourstackitem @\endcsname\relax%
        \else%
            \advance\contourmarkcount by1\relax%
            \let\currentcontourheight=\@contourstackitem%
            \contourmarkcode%
            \def\contourmarkstart{\contourmarkprefix-\the\contourmarkcount.west}%
            \def\contourmarkend{\contourmarkprefix-\the\contourmarkcount.east}%         
            \edef\contourpath{\contourpath \csname contourcontourpathcommand@\@contourstackitem @\endcsname}%
        \fi%
        \let\@next=\@@contourdolist%
        \let\@contourstacklastitem=\@contourstackitem%
    \fi
    \@next%
}

% \contourcontourpathcommand{<symbol>}{<contour path command code>}
% \contourmarkstart and \contourmarkend are setup as the
% left and right points of the charactor at zero contour height.
\def\contourcontourpathcommand#1{\expandafter\def\csname contourcontourpathcommand@#1@\endcsname}

% \contourmark{<symbol>}{<mark start height>}{<mark end height>}

\def\contourmark#1#2#3{%
    \contourcontourpathcommand{#1}{([shift={(0,#2)}]\contourmarkstart) -- ([shift={(0,#3)}]\contourmarkend)}
}

\makeatother

% Set up pitchmarks for inline and multiline depiction of tonal pitchlevels. 
% Other characters can be used to create a pitchmark, eg: < > ( ) - _ = '

\contourcontourpathcommand{.}{}

\contourmark{0}{0}{0}
\contourmark{1}{1}{1}
\contourmark{2}{2}{2}
\contourmark{3}{3}{3}
\contourmark{4}{4}{4}
\contourmark{5}{5}{5}
\contourmark{+}{0}{1}
\contourmark{,}{0}{2}
\contourmark{>}{0}{4}

% Sample pitchmarks
% Rise and fall
\contourcontourpathcommand{!}{
    (\contourmarkstart) .. controls ++(0,2) and ++(0,2) .. (\contourmarkend)
}

% Fall and rise - shift specifies the vertical height of the startpoint and endpoint
\contourcontourpathcommand{@}{
    ([shift={(0,2)}]\contourmarkstart) .. controls ++(0,-2) and ++(0,-2) .. ([shift={(0,2)}]\contourmarkend)
}

% Glide up (concave)
\contourcontourpathcommand{?}{
    (\contourmarkstart) .. controls ++(0,-1) and ++(0,-1) .. ([shift={(0,2)}]\contourmarkend)
}

% Glide up (convex)
\contourcontourpathcommand{:}{
    (\contourmarkstart) .. controls ++(0,1) and ++(0,1) .. ([shift={(0,2)}]\contourmarkend)
}

% Glide down (concave)
\contourcontourpathcommand{;}{
    ([shift={(0,2)}]\contourmarkstart) .. controls ++(0,-1) and ++(0,-1) .. (\contourmarkend)
}

% Glide down (convex)
\contourcontourpathcommand{^}{
    ([shift={(0,2)}]\contourmarkstart) .. controls ++(0,1) and ++(0,1) .. (\contourmarkend)
}

% Continue previous path
\contourcontourpathcommand{|}{
    -- ([shift={(0, 3)}]\contourmarkend)
}



% 
% Copyright Mark Wibrow 2013.
% 
% This program is free software: you can redistribute it and/or modify
% it under the terms of the GNU General Public License or the GNU
% Affero General Public License as published by the Free Software
% Foundation, either version 3 of the License, or (at your option)
% any later version.
% 
% This program is distributed in the hope that it will be useful,
% but WITHOUT ANY WARRANTY; without even the implied warranty of
% MERCHANTABILITY or FITNESS FOR A PARTICULAR PURPOSE.  See the
% GNU General Public License for more details.
% 
% You should have received a copy of the GNU General Public License
% and the GNU Affero General Public License along with this program.
% If not, see <http://www.gnu.org/licenses/>.

\newdimen\contourraise
\newdimen\contourspacetokenwidth
\newdimen\contourtokenkern
\newcount\lasttokennumber
\newcount\currenttokennumber
\newcount\contourmarkcount
\newcount\contourtokenunderlinestate
\newbox\contourbox
\makeatletter

\tikzset{
    tight fit/.style={
        inner sep=0pt,
        outer sep=0pt,
    },
    %
    %
    % How far above the reference anchor of the text,
    contour raise/.code=\pgfmathsetlength\contourraise{#1},
    contour reference anchor/.store in=\contourreferenceanchor,
    contour reference anchor=base east,
    % The `scale' for the values in the contour height specification
    contour scale/.store in=\contourscale,
    contour scale=3pt,
    % The prefix for the contour marks.
    contour mark prefix/.store in=\contourmarkprefix,
    contour mark prefix=contour,
    % The style for the contour path
    contour/.style={
        draw, 
        rounded corners=1ex,
    },
    % The style for the token nodes
    every contour token/.style={
        anchor=base west, 
        tight fit,
    },
    contour underline/.style={
        draw
    },
    % The character to insert a mark (use with care)
    contour mark character/.store in=\contourmarkchar,
    contour mark character=|,
    % Want to change the code for contour marks? Use this key.
    contour mark code/.store in=\contourmarkcode,
    % Want to change the code for tokens? Use this key.
    contour token code/.store in=\contourtokencode,
    % Want to change the code for drawing the contour? Use this  key.
    contour code/.store in=\contourcode,
    %
    % Default stuff
    contour mark code={%
        \coordinate (\contourmarkprefix-\the\contourmarkcount)
          at ([yshift=\contourraise, y=\contourscale,               
          shift={(0,\currentcontourheight)}]token-\the\currenttokennumber.\contourreferenceanchor);
    },
    contour token code={%
        \node [every contour token/.try] at 
        ([xshift=\contourtokenkern]token-\the\lasttokennumber.base east) 
            (token-\the\currenttokennumber) {\token};
    },
    contour code={
        \draw [contour] (\contourmarkprefix-1)
            \foreach \y in {2,...,\the\contourmarkcount}{ -- 
                    (\contourmarkprefix-\y) };                  
    },
    contour marks/.style={
        contour mark list={#1},
        contour code={
             \draw [y=\contourscale, contour] \contourpath;                  
         },
         contour mark code={%
            \coordinate (@a) at ([yshift=\contourraise]token-\the\currenttokennumber.base west);
            \coordinate (@b) at ([yshift=\contourraise]token-\the\currenttokennumber.base east);
            \node [tight fit, fit={(@a) (@b)}] (\contourmarkprefix-\the\contourmarkcount) {};
        },
    },
    % Don't draw the contour.
    tokens only/.style={
        contour code={}
    },
    %
    % Only draw the contour (but the space is still used for the tokens)
    contour only/.style={
        every contour token/.append style={
            execute at begin node={\setbox\contourbox=\hbox\bgroup},
            execute at end node=\egroup\phantom{\box\contourbox}%
        },
        underline/.style={
            draw=none
        }
    },
    %
    % Make tokens follow the contour marks.
    tokens follow contour/.style={
        tokens only,
        contour token code={%
            \node [every contour token/.try, y=\contourscale] at 
                ([xshift=\contourtokenkern]token-\the\lasttokennumber.base east |- 
                0,\currentcontourheight) 
                (token-\the\currenttokennumber) {\token};
        },
    },
    % What style to use when drawing underline
    underline/.style={
        draw
    },
    % The underline is drawn along the south side of a node which 
    % takes this style.
    underline token/.style={
        inner ysep=1pt
    },
    % When grouping tokens (e.g., for putting box around)
    % this style is applied to a node that is fitted around the group
    token group/.style={
        inner xsep=1pt,
        inner ysep=2pt,
        rounded corners=2pt
    },
    % Draw boxes around tokens groups.
    box tokens/.style={
        token group/.append style={
            draw
        }
    },  
    % Change the width of the spaces.
    space token width/.code=\pgfmathsetlength\contourspacetokenwidth{#1},
    space token width=0.125cm,
    contour mark list/.store in=\@contourmarklist%
}

\def\at@{@}

\let\@contourmarklist=\@empty

\def\contour{%
    \pgfutil@ifnextchar[{\contour@opts}{\contour@opts[]}}
\def\contour@opts[#1]{%
    \pgfutil@ifnextchar x{\contour@@opts[#1]}{\contour@@opts[#1]}}
\def\contour@@opts[#1]#2;{%
    \begin{scope}[#1]
        \coordinate (token-0);
        \currenttokennumber=0\relax%
        \lasttokennumber=0\relax%
        \contourmarkcount=0\relax%
        \def\lastcontourheight{0}%
        \contourtokenunderlinestate=0\relax%
        \let\lastcontourtoken=\relax%
        \contourtokenkern=0pt\relax%
        \def\contourpath{}%
        \@contour#2@%
}

% Must check for spaces
\def\@contour{\futurelet\@token\@checkforspace}

\def\@uscore{_}
\def\@checkforspace{%
    \ifx\@token\pgfutil@sptoken%
        \let\@next=\@replacespace%
    \else%
        \if\@token\contourmarkchar%
            \let\@next=\@contour@insertmark
        \else%
            \if\@token\@uscore
                \let\@next=\@contourtoggleunderline%
            \else%
                \let\@next=\@@contour%
            \fi%
        \fi%
    \fi%
    \@next%
}

\def\@contourtoggleunderline#1{%
    \advance\contourtokenunderlinestate by1\relax
    \ifnum\contourtokenunderlinestate>3\relax%
        \contourtokenunderlinestate=0\relax%
    \fi%
    \@contour%
}

\def\@contour@insertmark{%
    \afterassignment\@@contour@insertmark\let\@token=%
}

\def\@@contour@insertmark{%
    \futurelet\@token\@@@contour@insertmark}%

\def\@@@contour@insertmark{%
    \if\@token[%
        \let\@next=\@@@@contour@insertmark%
    \else%
        \let\currentcontourheight=\lastcontourheight%
        \let\@next=\@@@@@contour@insertmark%
    \fi%
    \@next%
}

\def\@@@@contour@insertmark[#1]{%
    \def\@tmp{#1}%
    \ifx\@tmp\@empty%
        \let\currentcontourheight=\lastcontourheight%
    \else%
        \def\currentcontourheight{#1}%
    \fi%
    \@@@@@contour@insertmark}

\def\@@@@@contour@insertmark{%
    \advance\contourmarkcount by1\relax%
     % Code for inserting mark
    \contourmarkcode%
    \let\lastcontourheight=\currentcontourheight%
    \@contour}

\def\contourspacetoken{{\hbox to \contourspacetokenwidth{\hfill}}}

\def\@replacespace#1{%
    \@contour\contourspacetoken#1%
}

\def\@@countour@afterlatenode{%
    \pgf@x=\pgfpositionnodelatermaxx\relax%
    \advance\pgf@x by-\pgfpositionnodelaterminx\relax%
    \global\edef\@contournodewidth{\the\pgf@x}%
}

\def\@@contour#1{%
    \def\@token{#1}%
    \if\@token\at@%
        \@contourdounderline%
        \pgfutil@ifundefined{pgf@sh@ns@tokengroup}{}{%
            \node [tight fit, fit={(tokengroup)}, token group/.try] {};
            \global\let\pgf@sh@ns@tokengroup=\relax%
        }%
        \let\@next=\@@@contour%
    \else%
        \lasttokennumber=\currenttokennumber%
        \advance\currenttokennumber by1%
        \contourtokenkern=0pt\relax%
        \ifnum\currenttokennumber>1\relax%
            %
            % Take care of kerning.
            % 
            % First get the width of the last and current token in the same hbox.
            %
            \let\pgfpositionnodelaterbox=\contourbox
            \pgfpositionnodelater\@@countour@afterlatenode%
            \def\token{\lastcontourtoken\@token}%
            \begingroup%
                \tikzset{every contour token/.append style={tight fit}}%
                \contourtokencode%
            \endgroup%
            \let\@contourkerntmp=\@contournodewidth%
            % 
            % Now subtract the width of last and current token in separate boxes.
            %
            \def\token{\hbox{\lastcontourtoken}\hbox{\@token}}%
            \begingroup%
                    \tikzset{every contour token/.append style={tight fit}}%
                    \contourtokencode%
            \endgroup%
            \pgfmathsetlength\contourtokenkern{\@contourkerntmp-\@contournodewidth}%
            \pgfpositionnodelater\relax%
        \fi%
        %
        % OK, now actually typeset the current token
        %
        \let\token=\@token%
        \contourtokencode%
        \let\lastcontourtoken=\token%
        % Manage underline state
        \@contourdounderline%
        \def\@@token{\contourspacetoken}%
        \ifx\@token\@@token%
            \pgfutil@ifundefined{pgf@sh@ns@tokengroup}{}{%
                \pgfutil@ifundefined{pgf@sh@ns@underline}{}{%
                    \node [tight fit, fit={(tokengroup) (underline)}] 
                    (tokengroup) 
                {};}%
                \node [tight fit, fit={(tokengroup)}, token group/.try] {};
                \global\let\pgf@sh@ns@tokengroup=\relax%
            }%
        \else
            \pgfutil@ifundefined{pgf@sh@ns@tokengroup}{%
                \node [tight fit, 
                fit={(token-\the\currenttokennumber)}] 
                (tokengroup) {};
            }{%
                \node [tight fit, 
                fit={(token-\the\currenttokennumber) 
                (tokengroup)}] 
                (tokengroup){};
            }%
        \fi%
        \let\@next=\@contour
        %
    \fi%
    \@next%
}

\def\@contourdounderline{%
    \ifcase\contourtokenunderlinestate%
     \or
         \node [tight fit, fit={(token-\the\currenttokennumber)}] 
         (underline) {};
         \contourtokenunderlinestate=2\relax%
     \or%
            \node [tight fit,fit={(token-\the\currenttokennumber) (underline)}]
            (underline) {};
     \or%
            \node [tight fit, fit={(underline)}, underline token/.try] 
            (underline) {};
         \draw [underline/.try]
                    (underline.south west) -- (underline.south east);
            \pgfutil@ifundefined{pgf@sh@ns@tokengroup}{}{%
                 \node [tight fit, fit={(tokengroup) (underline)}] 
                 (tokengroup) {};%
                 \node [tight fit, fit={(tokengroup)}, token group/.try] {};
                 \global\let\pgf@sh@ns@tokengroup=\relax%
                 \global\let\pgf@sh@ns@underline=\relax%
             }
         \contourtokenunderlinestate=0\relax
     \fi%
}
\def\@@@contour{%
    \ifx\@contourmarklist\@empty%
    \else%
        \@contourdolist%
    \fi%
    \ifnum\contourmarkcount>1
        % Code for drawing contour
        \contourcode%
    \fi%
    \end{scope}%
    \ignorespaces%
}

\def\@contourstackpop{%
    \let\@contourstackitem=\@empty%
    \ifx\@contourstack\@empty%
    \else%
        \expandafter\@@contourstackpop\@contourstack\@@contourstackpop%
    \fi%
}

\def\@@contourstackpop#1#2\@@contourstackpop{%
    \def\@contourstackitem{#1}%
    \ifx\@contourstackitem\@empty%
        \def\@contourstackitem{#2}%
        \let\@contourstack=\@empty%
    \else%
        \def\@contourstack{#2}%
    \fi%
}

\def\@contourdolist{%
    \let\@contourstack=\@contourmarklist%
    \let\@contourstacklastitem=\@empty%
    \let\contourpath=\@empty%
    \edef\contourtotaltokens{\the\currenttokennumber}%
    \currenttokennumber=0\relax%
    \contourmarkcount=0\relax%
    \@@contourdolist%
}

\def\@@contourdolist{%
    \@contourstackpop%
    \advance\currenttokennumber by1\relax%
    \ifx\@contourstackitem\@empty%
        \let\@next=\relax%
    \else%
        \expandafter\ifx\csname contourcontourpathcommand@\@contourstackitem @\endcsname\relax%
        \else%
            \advance\contourmarkcount by1\relax%
            \let\currentcontourheight=\@contourstackitem%
            \contourmarkcode%
            \def\contourmarkstart{\contourmarkprefix-\the\contourmarkcount.west}%
            \def\contourmarkend{\contourmarkprefix-\the\contourmarkcount.east}%         
            \edef\contourpath{\contourpath \csname contourcontourpathcommand@\@contourstackitem @\endcsname}%
        \fi%
        \let\@next=\@@contourdolist%
        \let\@contourstacklastitem=\@contourstackitem%
    \fi
    \@next%
}

% \contourcontourpathcommand{<symbol>}{<contour path command code>}
% \contourmarkstart and \contourmarkend are setup as the
% left and right points of the charactor at zero contour height.
\def\contourcontourpathcommand#1{\expandafter\def\csname contourcontourpathcommand@#1@\endcsname}

% \contourmark{<symbol>}{<mark start height>}{<mark end height>}

\def\contourmark#1#2#3{%
    \contourcontourpathcommand{#1}{([shift={(0,#2)}]\contourmarkstart) -- ([shift={(0,#3)}]\contourmarkend)}
}

\makeatother

% Set up pitchmarks for inline and multiline depiction of tonal pitchlevels. 
% Other characters can be used to create a pitchmark, eg: < > ( ) - _ = '

\contourcontourpathcommand{.}{}

\contourmark{0}{0}{0}
\contourmark{1}{1}{1}
\contourmark{2}{2}{2}
\contourmark{3}{3}{3}
\contourmark{4}{4}{4}
\contourmark{5}{5}{5}
\contourmark{+}{0}{1}
\contourmark{,}{0}{2}
\contourmark{>}{0}{4}

% Sample pitchmarks
% Rise and fall
\contourcontourpathcommand{!}{
    (\contourmarkstart) .. controls ++(0,2) and ++(0,2) .. (\contourmarkend)
}

% Fall and rise - shift specifies the vertical height of the startpoint and endpoint
\contourcontourpathcommand{@}{
    ([shift={(0,2)}]\contourmarkstart) .. controls ++(0,-2) and ++(0,-2) .. ([shift={(0,2)}]\contourmarkend)
}

% Glide up (concave)
\contourcontourpathcommand{?}{
    (\contourmarkstart) .. controls ++(0,-1) and ++(0,-1) .. ([shift={(0,2)}]\contourmarkend)
}

% Glide up (convex)
\contourcontourpathcommand{:}{
    (\contourmarkstart) .. controls ++(0,1) and ++(0,1) .. ([shift={(0,2)}]\contourmarkend)
}

% Glide down (concave)
\contourcontourpathcommand{;}{
    ([shift={(0,2)}]\contourmarkstart) .. controls ++(0,-1) and ++(0,-1) .. (\contourmarkend)
}

% Glide down (convex)
\contourcontourpathcommand{^}{
    ([shift={(0,2)}]\contourmarkstart) .. controls ++(0,1) and ++(0,1) .. (\contourmarkend)
}

% Continue previous path
\contourcontourpathcommand{|}{
    -- ([shift={(0, 3)}]\contourmarkend)
}


